\documentclass[a4paper]{article} % \documentclass{} is the first command in any LaTeX code.  It is used to define what kind of document you are creating such as an article or a book, and begins the document preamble
\usepackage[dvipsnames]{xcolor}
\usepackage[a4paper,width=150mm,top=25mm,bottom=25mm]{geometry}
\usepackage{fancyhdr}
\usepackage{paralist}
\usepackage{tikz}
\usepackage{enumitem}
\usepackage{caption}  
\usetikzlibrary{calc}
\setlength{\headheight}{34.54448pt}
\pagestyle{myheadings}
\fancypagestyle{plain}{
    \lhead{John David Vidad\\2019--10432\\CS 145 LAB 3}
    \rhead{}
}
\renewcommand{\headrulewidth}{0pt}
\renewcommand{\footrulewidth}{0pt}
\title{CS 145 Project\:Parameter-Adaptive Reliable UDP-based
Protocol}
\author{}
\date{}
\begin{document} % All begin commands must be paired with an end command somewhere
    \maketitle
    \section{Requirements}
    \begin{itemize}
        \item \textbf{Level Declaration:} Level 3
    \end{itemize}
    \section{Documentation}
    \subsection{Code Explanation}
    The code is divided into 5 major parts: 
    the possible changing of parameters, 
    the getting of transaction ID,
    the estimation of payload size,
    the adjustment of the estimated of payload size, and
    the sending of the remaining payload
    \subsubsection{changing of parameters}
    With the assumption of valid inputs, we traverse the arguements by pair after the python file arguement.
    \begin{center}
        \begin{tikzpicture}
            \node[above right,
            inner sep=0](image) at (0,0) {\includegraphics[width=0.9\textwidth]{parameters.png}};
            \begin{scope}[
                x={($0.1*(image.south east)$)},
                y={($0.1*(image.north west)$)}]
                \draw[very thick,white] (0.7,6.2) rectangle (6,8.5)
                node[black,fill=white]{\small default values};
                \draw[very thick,red] (0.75,5.3) rectangle (8,5.9)
                node[white,fill=red]{\small Traversing by pair};
                \draw[very thick,blue](1.2,4.4)rectangle(7.4,5.25)
                node[white,fill=blue]{\small Payload Changing};
                \draw[very thick,magenta](1.2,3.5)rectangle(7.4,4.35)
                node[white,fill=magenta]{\small IP of receiver changing};
                \draw[very thick,cyan](1.2,2.6)rectangle(7.4,3.45)
                node[white,fill=cyan]{\small changing port of receiver};
                \draw[very thick,orange](1.2,1.65)rectangle(7.4,2.55)
                node[white,fill=orange]{\small changing port of sender};
                \draw[very thick,Melon](1.2,0.7)rectangle(7.4,1.65)
                node[white,fill=Melon]{\small changing id};
            \end{scope}
        \end{tikzpicture}
    \captionof{figure}{parameter changing snippet}
    \end{center}
    Running a python file will have at least 1 arguement and via the argv of the sys module
    we are able to access it as list of strings with the first index being the python file.
    To show that this is working, we simply created a show input 
    wherein it shows all the variables like ip of the receiver, the port numbers,
    the id, and he period.
    below is a Screenshot shot of testing and showing that the changing of parameters work.
    \begin{center}
        \begin{tikzpicture}
            \node[above right,
            inner sep=0](image) at (0,0) {\includegraphics[width=\textwidth]{parameter test.png}};
            \begin{scope}[
                x={($0.1*(image.south east)$)},
                y={($0.1*(image.north west)$)}]
                
                \draw[very thick,Purple] (0,9.45) rectangle (9.9,9.9);
                \draw[latex-, very thick,Purple] (6,9.45) -- ++(0.3,-0.2)
                node[below,white,fill=Purple]{\small Changing all Parameters};
                \draw[very thick,ProcessBlue] (0,6.9) rectangle (7,7.5);
                \draw[latex-, very thick,ProcessBlue] (7,7.2) -- ++(0.3,0)
                node[right,white,fill=ProcessBlue]{\small default parameters};
            \end{scope}
        \end{tikzpicture}
    \captionof{figure}{parameter changing test}
    \end{center}
    Each Test case either changes the default values on indication or
    stay with the default, for a clearer view, all figures will be viewable via a google drive link provided in the Links section with 
    names in the figure number.
\section{Links}
\begin{itemize}
    \item \item \textbf{GitHub Link:} Level 3
\end{itemize}
\end{document}